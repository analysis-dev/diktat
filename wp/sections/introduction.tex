It is necessary to conform to a specific style of code during software development, otherwise it will reduce the ability to better understand programmers’ intent and find more functional defects. Analyzers, in turn, have methods for finding and correcting style errors. 

Static code analysis is useful not only for optimization and increasing effectiveness but also for automatic error detection.

There are many methods and techniques used by existing analyzers to find bugs (path-sensitive data flow analysis \cite{ref:kremenek}, alias analysis \cite{ref:effective}, type analysis \cite{ref:simple}, symbolic execution \cite{ref:dis}, abstract interpretation \cite{ref:dis}.

Static analysis can be thought of as an automated code review process. Of the tasks solved by static code analysis programs, two main ones can be distinguished: identifying errors in programs and recommending code formatting. That is, the analyzer allows you to check whether the source code complies with the accepted coding standard. Also, a static analyzer can be used to determine the maintainability of a code, which is how easy it is to analyze, modify and adapt a given software. Static analysis tools allow you to identify a large number of errors in the design phase, which significantly reduces the development cost of the entire project. Static analysis covers the entire code - it checks even those code fragments that are difficult to test. It does not depend on the compiler used and the environment in which the compiled program will be executed.