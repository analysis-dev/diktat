\subsection{What is diKTat?}
DiKTat \footnote{\url{https://github.com/cqfn/diKTat}} - is a formal strict code-style for Kotlin language and a linter with a set of rules that implement this code-style. Basically, it is a collection of Kotlin code style rules implemented as AST visitors on top of KTlint framework \footnote{\url{https://github.com/pinterest/ktlint}}. Diktat detects and automatically fixes code style errors and code smells based on the configuration of rules. DiKTat is a highly configurable framework, that can be extended further by adding custom rules. It can be run as command line application or with maven or gradle plugins. In this paper, we will explain how diKTat works, describe its advantages and disadvantages and compare it with other static analyzers for Kotlin. The main idea is to use diktat in your CI/CD pipeline.

\subsection{Why diKTat?}
DiKTat permits formal flexible description or Rules and Inspections expressed by means of yml file. We looked at similar existing projects and realized that their functionality does not give us a chance to implement our own configurable code style. Most of rules which we wanted to implement were missing in other analyzers. Mostly all of those analyzers had hardcoded logic and prohibited configuration. That’s why we decided that we need to create convenient, user friendly and easily configured tool for developers.

First of all, diKTat has its own highly configurable Ruleset $R_{diktat}$ that contains unique Inspections, missing in other Kotlin static analyzers. You just need to set your own options which fit your project the most. In case you don't want to do this - you can use the default configuration, but some of complex inspections will be disabled. Basically, Ruleset is an \texttt{yml} file with a description of each rule.

Secondly, DiKTat has its own plugins and can be run via Maven, Gradle and command line. Developer can use build automation system that he prefers.

Finally, developer can disable with diKTat each inspection from the code using special annotations on the line where he wants to suppress an Inspection.