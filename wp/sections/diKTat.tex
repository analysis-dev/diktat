\subsection{What is diKTat?}
DiKTat - is a formal strict code style (\url{https://github.com/cqfn/diKTat}) and a linter with a set of rules that implement this code style for Kotlin language. Basically, it is a collection of Kotlin code style rules implemented as AST visitors on top of KTlint framework (\url{https://github.com/pinterest/ktlint}). Diktat warns and fixes code style errors and code smells based on configuration file. DiKTat is a highly configurable framework, that can be extended further by adding custom rules. It can be run as command line application or with maven or gradle plugins. In this paper, we will explain how DiKTat works, describes advantages and disadvantages and how it differs from other static analyzers.

\subsection{Why diKTat?}
So why did we decide to create diKTat? We looked at similar projects and realized that they have defects and their functionality does not give you a chance to implement modern configurable code style. That’s why we came to a conclusion that we need to create convenient and easy-to-use tool for developers. Why is it easy-to-use? First of all, diKTat has its own highly configurable ruleset. You just need to fill your own options on rules in ruleset or either use default one. Basically, ruleset is an yml file with a description of each rule. Secondly, there are a lot of developers that use different tools for building projects. Most popular are Maven and Gradle. DiKTat supports these ones and it also has cli. Finally, each developer has their own codestyle and sometimes they don’t want static analyzers to trigger on some lines of code. In diKTat you can easily disable a rule.