\subsection{What is diKTat?}
DiKTat \footnote{\url{https://github.com/cqfn/diKTat}} - is a formal strict code-style for Kotlin language and a linter with a set of rules that implement this code-style. Basically, it is a collection of Kotlin code style rules implemented as AST visitors on top of KTlint framework \footnote{\url{https://github.com/pinterest/ktlint}}. Diktat detects and automatically fixes code style errors and code smells based the configuration of rules. DiKTat is a highly configurable framework, that can be extended further by adding custom rules. It can be run as command line application or with maven or gradle plugins. In this paper, we will explain how diKTat works, describe it's advantages and disadvantages and compare it with other static analyzers for Kotlin. The main idea is to use diktat in your CI/CD pipeline.

\subsection{Why diKTat?}
So why did we decide to create diKTat? We looked at similar existing projects and realized that their functionality does not give us a chance to implement our own configurable code style. Most of rules that we wanted to implement were missing in other analyzers. Mostly all of those analyzers had hardcoded logic and prohibited configuration. That’s why we decided that we need to create convenient and user friendly tool for developers that can be easily configured. Why is it easy-to-use?

First of all, diKTat has its own highly configurable Ruleset $R_{diktat}$ that contains unique Inspections that are missing in other Kotlin static analyzers. You just need to set your own options that fit your project the most. In case you don't want to do this - you can use the default configuration, but some of complex inspections will be disabled. Basically, Ruleset is an \texttt{yml} file with a description of each rule. 

Secondly, diKTat has it's own plugins and can be run via Maven, Gradle and command line. Developer can use build automation system that he prefers.

Finally, with diKTat developer can disable each inspection from the code using special annotations on the line where he wants to suppress an Inspection.