\subsection{CLI-application}
\par You can run diKTat as a CLI-application by installing it first. You can find detailed instructions on how to install and run on different OS on github (\url{https://github.com/cqfn/diKTat/blob/master/README.md#run-as-cli-application}). After the run, errors will be found and displayed. Each error consists of a rule name, a description of the rule so that the user can understands the error, the line and column number where the error was found.\\
\subsection{Plugins}
\par Alternatively, you can add a diktat maven or gradle plugin directly to the project: detailed instructions for maven can be found here - \url{https://github.com/cqfn/diKTat/blob/master/README.md#run-with-maven}and for gralde - \url{https://github.com/cqfn/diKTat/blob/master/README.md#run-with-gradle-plugin}.\\
\subsection{Configuratifon file}
As described above, diKTat has a configuration file. Note that you should place the \textsl{diktat-analysis.yml} file containing the diktat configuration in the parent directory of your project when running as a CLI application. Diktat-maven-plugin and diktat-gradle-plugin have a separate option for configuration file path.
\subsection{WEB}
\par Of course, the easiest way to use it without any downloads or installations is the web version of the app. You can try it by following the link \url{https://ktlint-demo.herokuapp.com/demo}. Web app supports both checking and fixing, using either ktlint or diktat ruleset. For diktat you can also upload a custom configuration file.
